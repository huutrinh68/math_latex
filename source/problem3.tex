\section*{Bài 3:}
Giải phương trình:
\begin{enumerate}[label=(\alph*)]
\item $(x^2 - x - 2)\sqrt{x+1} = 0$
\item $\frac{x^2}{\sqrt{x-2}}=\frac{1}{\sqrt{x-2}} - \sqrt{x-2}$
\end{enumerate}

\subsection*{Lời giải:}
\begin{enumerate}[label=(\alph*)]
\item Điều kiện xác định: $x \geq -1$ \\
Ta có $x=-1$ là một nghiệm. \\
Nếu $x > 1$ thì $\sqrt{x+1} > 0$. \\
Do đó phương trình tương  $x^2 - x -2 = 0 \Leftrightarrow x=-1$ hoặc $x=2$ \\
Đối chiếu điều kiện ta được nghiệm của phương trình là $x = -1, x = 2$. \\
Vậy phương trình đã cho có hai nghiệm S = $\{-1; 2\}$
\item Điều kiện xác định: $x > 2$ \\
 Với điệu kiện đó phương trình tương đương với phương trình:
\begin{eqnarray*}
x^2 &=& 1 - (x-2) \\
x^2 + x -3 &=& 0 \\
x &=& \frac{-1 \pm \sqrt{13}}{2}
\end{eqnarray*}
Đối chiếu với điều kiện ta thấy không có giá trị nào thỏa mãn. \\
Vậy phương trình vô nghiệm
\end{enumerate}