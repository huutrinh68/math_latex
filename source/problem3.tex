\section*{Bài 3:}
Giải phương trình:
\begin{enumerate}[label=(\alph*)]
\item $(x^2 - x - 2)\sqrt{x+1} = 0$
\item $\frac{x^2}{\sqrt{x-2}}=\frac{1}{\sqrt{x-2}} - \sqrt{x-2}$
\item $x + \frac{1}{x-1} = \frac{2x-1}{x-1}$
\item $1 + \frac{1}{x-3} = \frac{5}{x^2-x-6}$
\end{enumerate}

\subsection*{Lời giải:}
\begin{enumerate}[label=(\alph*)]
\item Điều kiện xác định: $x \geq -1$ \\
Ta có $x=-1$ là một nghiệm. \\
Nếu $x > 1$ thì $\sqrt{x+1} > 0$. \\
Do đó phương trình tương  $x^2 - x -2 = 0 \Leftrightarrow x=-1$ hoặc $x=2$ \\
Đối chiếu điều kiện ta được nghiệm của phương trình là $x = -1, x = 2$. \\
Vậy phương trình đã cho có hai nghiệm S = $\{-1; 2\}$

\item Điều kiện xác định: $x > 2$ \\
 Với điệu kiện đó phương trình tương đương với phương trình:
\begin{eqnarray*}
x^2 &=& 1 - (x-2) \\
x^2 + x -3 &=& 0 \\
x &=& \frac{-1 \pm \sqrt{13}}{2}
\end{eqnarray*}
Đối chiếu với điều kiện ta thấy không có giá trị nào thỏa mãn. \\
Vậy phương trình vô nghiệm

\item Điều kiện $x \neq 1$ \\
Với điều kiện trên phương trình tương đương với:\\
\begin{eqnarray*}
x^2 - x + 1 = 2x -1 \\
x = 1 \quad \textrm{or} \quad x = 2
\end{eqnarray*}
Đối chiếu điều kiện ta được phương trình có nghiệm duy nhất $x = 2$.

\item Điều kiện xác đinh:
\begin{displaymath}
\left\{ \begin{array}{l}
x \neq 3 \\
x^2 -x -6 \neq 0
\end{array} \right.  \; \Leftrightarrow \; \left\{ \begin{array}{l}
x \neq 3 \\
x \neq -2
\end{array} \right.
\end{displaymath}
Với điều kiện đó phương trình tương đương với\\
\begin{eqnarray*}
1 + \frac{1}{x-3} &=& \frac{5}{(x-3)(x+2)} \\
(x-3)(x+2) + x + 2 &=& 5 \\
x^2 &=& 9 \\
x &=& \pm 3
\end{eqnarray*}
Đối chiếu với điều kiện ta có nghiệm của phương trình là $x = -3$
\end{enumerate}